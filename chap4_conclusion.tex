\chapter{Outlook}

\section{Lessons Learned}

To arrive at a truly predictive, multiscale model of disordered organic systems, structural assumptions and hypothesizes need to be quantitatively confirmed. Physical intuition helps guide research, but even with ample expertise it is e.g. difficult to discern features in a torsion potential from the visual inspection of the molecule, or conversely, extrapolate polymer structure from looking at the torsion potential. Throughout the presented work there are a number of examples where our results were not intuitive. The torsion potential of bithiophene (BT)/polythiophene (PT) is a good example (Fig.~\ref{fig:comp_tor}, Fig.~\ref{fig:a_vs_c}). Considering the atomic structure of BT, one might assume the minimum energy torsional configuration is trans (180\textdegree) planar due to the lack of bulky side groups. However, the minimum energy configuration is found to be non-planar. Knowing this, it is conceivable that there is steric interaction between the hydrogens causing the non-planarity. In reality, we demonstrate that the drive for ring aromaticity is largely responsible for the non-planar minimum. Another example of unexpected polymer structure was the chain length of PT as a function of excitation. Based on the excited-state torsion potential (Fig.~\ref{fig:comp_tor}), which has two planar minima, one may assume a longer chain end-to-end distance. As shown in Fig.~\ref{fig:ideal} the cis (0\textdegree) planar configuration has a large impact on chain structure allowing a chain to get shorter while remaining planar. In summary, chemical intuition for torsion potentials and chain structure of disordered conjugated polymers is often misleading and we highly recommend using representative examples and visualization whenever possible.

The level of theory and basis set used for quantum calculations on conjugated systems should be carefully considered. The recommendations here are intended for gaussian-type orbital quantum calculations. For neutral ground-state torsional calculations, qualitative agreement can be found for a variety of methods assuming a sufficiently large basis set is employed. We routinely use def2-TZVPP(D) triple and def2-QZVPP(D) quadruple-zeta basis sets as they balance accuracy and computational efficiency, and usually do not require a complete basis set extrapolation as do Dunning basis sets. For large extended systems we apply the double-zeta 6-31++G** Pople basis set because of its efficiency.

Doped and excited-state systems require proper carrier localization are considerably more sensitive to the level of quantum chemistry theory employed as compared to the neutral ground-state. Hartree-Fock (HF) methods over localize, whereas density functional theory (DFT) methods (e.g. LDA and B3LYP) unphysically delocalize the carrier [REF]. As a result, hybrid functionals that mix HF electron-exchange and DFT electron-correlation are usually recommended for conjugated polymers as they balance localization and are computationally tractable for larger systems [REF]. In this work, we have successfully employed two hybrid DFT functionals; wB97X-D and wB97M-V. Higher level methods such as second order Moller-Plesset theory (MP2) can accurately describe carrier localization, but in our experience MP2 calculations exhibit considerable spin contamination for certain conjugated systems and are more computationally demanding.

Quantum chemistry methods for excited conjugated systems are limited and new approaches are desirable. In this work doped systems are modeled as cations which can be accurately captured by either restricted open shell (RO) or unrestricted open shell (UO) DFT. Excited-states on the other hand generally require different methods such as time-dependent DFT (TDDFT). The only exception is the lowest lying triplet (T1) which can be described with RO or UO-DFT---an exception we took advantage of in Chapter 2. In our experience, standard TDDFT did not produce quality results, and it would be highly advantageous if a functional tuning procedure was developed that allowed quantitative results from TDDFT. Although a number of such procedures exist [REF], they are cumbersome to implement, and we had little success using them. Additionally, the tuning procedure should be automated since it would need to be repeated for each unique molecule. Quantitative results from TDDFT would enable solvation approximations (as discussed below) because many modern TDDFT codes include implicit solvation models.

\section{Future Work}

\subsection{Torsion Potential Model}

There are a variety of ways the torsion potential model presented in Chapter 2 can be extended. We list a number of ideas below.

\begin{itemize}
  \item Consideration of carrier (polaron/exciton) delocalization

  In the current algorithm, doped and excited torsion angles are considered individually. A potentially more physical picture would be a group of torsion angles. For instance, the length of a polaron delocalization in PT is between 5-10 rings [REF], which represents 4-9 torsion angles. Accordingly, groups of torsion angles that reflect the carrier delocalization length could be incorporated into the model. Torsion groups could still be randomly placed, until an interaction term is established. The change to groups may alter the persistence length results; however, it is unclear if S values would be affected due to the use of the director, which is akin to an average across the entire chain.

  \item Include the first singlet (S1) excitation state

  When the model system PT is photoexcited the lowest lying spin allowed state is S1, although the first triplet (T1) is lower in energy. Excitons can access the T1 state through an intersystem crossing. Currently, we have only considered the T1 state because we are focused on the steady-state behavior (i.e. the S1 state will have time to transition to the T1), and the T1 torsion potential can be calculated with ground-state DFT methods. Although we expect the torsion potential of the S1 and T1 states to be qualitatively similar it would be interesting if any quantitative differences are obtained.

  \item Include Solvation

  Solvent effects can have a large impact on the energetics of excited states [REF]. To the best of our knowledge, solvent effects on doped or excited torsion potentials for specific conjugated systems remains largely undetermined. It is an important point to consider because none of the systems modeled in this thesis exist in vacuum. The simplest way to include some solvent effects would be applying an implicit solvation model to the torsion potential quantum calculations.

  \item Analytical Solution

  Our torsion potential model is numerical, but it may be possible to produce an analytical solution for some properties of interest. If a polymer chain of torsion angles is described as a two-state system [REF], where one state is a doped or excited torsion and the other state is a ground-state torsion, we can write down the partition function Eq ??. Deriving the persistence length or S value from the partition function remains challenging. A good starting point for the persistence length is the rotational isomeric state (RIS) model [REF]. Overall, an analytic solution could provide improvements in computational efficiency compared to the numerical model.

\end{itemize}

\subsection{Aromaticity}

The impact of aromaticity in Chapter 3 has a few natural extensions and follow-up questions, which are discussed below.

\begin{itemize}
  \item Extending our methods to other systems

  It is worthwhile to investigate how aromaticity changes as a function of torsion angle for other noncovalent locking systems. For instance, systems where nontraditional hydrogen-bonding has been reported as the interaction responsible for planarity [REF]. Another system that would benefit from further investigation is bis-3,4-ethylenedithiothiophene (BEDTT). Replacing the pendant group oxygen with sulfur changes BEDOT to BEDTT, yet the torsion potentials are substantially different such that BEDTT exhibits a non-planar minimum [REF]. Preliminary aromaticity calculations performed on BEDTT were not immediately clarifying and require further work. Remarkably, if the pendant group S atoms are replaced with another chalcogen such as Se, the system reverts back to being planar [REF]. We recommend systematic study of chemical trends within the chalcogens at the electronic structure level to elucidate structural-chemistry design rules.

  \item Doped and excited-state analysis

  Aromaticity may help predict the conductivity of conjugated polymers. Electronic structure rearrangement occurs when conjugated polymers are doped and excited and accordingly the aromaticity changes. Understanding how aromaticity changes between the ground and doped or excited-state may lead to some revealing correlations. Others have made connections between aromaticity and conductivity as well [REF].

\end{itemize}
