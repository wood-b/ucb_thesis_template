% (This file is included by thesis.tex; you do not latex it by itself.)

\begin{abstract}

\setlength\parindent{20pt}
\setlength{\parskip}{0.08mm}
% The text of the abstract goes here.  If you need to use a \section
% command you will need to use \section*, \subsection*, etc. so that
% you don't get any numbering.  You probably won't be using any of
% these commands in the abstract anyway.

\noindent Characterizing structure-function relationships is of fundamental importance across science, from biology to condensed matter physics. These relationships provide deep insights that can lead to innovation, whether it is connecting the structure of a protein to its activity or understanding how crystalline defects alter the function of a material. Conjugated polymers are a class of plastics that can be electronically conductive, and have many potential applications due to their unique blend of physical and electronic properties. In this thesis, we focus on the atomic structure of conjugated organic systems and the effect it has on their electronic properties.

Carrier mobility in conjugated polymer materials is limited by the structure of amorphous chains that connect domains of varying crystallinity and orientation. Furthermore, for a wide range of conjugated polymers, it is established that doping and excitation induce torsional rearrangements. Nevertheless, little is known about the long-range impact these rearrangements have on chain structure. To further optimize carrier mobility in conjugated polymer materials, an improved understanding of doped and excited amorphous chain structure is necessary. We develop a multiscale model that captures the underlying electronic structure with torsion potentials which are then used to generate chain conformations as a function of doping or excitation. We confirm that the ground-state torsion potential minima are non-planar and that the minima shift to planar configurations in the doped and excited states. As a result, chain planarity monotonically increases with the level of doping or excitation, and carrier mobility is fundamentally connected to planarity. However, for the model system polythiophene, increasing planarity does not always correspond to more linear or longer chains. We find that the chain length trends diverge between doped and excited chains, despite exhibiting similar planarity. Our results offer structural insights for design strategies to tune electronic properties of aromatic conjugated polymers.

Our polymer model provides key information at the chain level, however, to reach the full design potential of functional organic materials it is essential to understand the specific interactions that govern the torsional structure of conjugated systems. Creating planar or ``locking'' molecular structures are of particular interest for tuning electronic properties. While the incorporation of noncovalent locks is an effective strategy for increasing planarity, the precise interactions leading to these planar structures are often unknown or mischaracterized. In this thesis, we demonstrate that aromaticity can be used to understand and interrupt the complex physical interactions which lead to planarity. We illustrate the important role aromaticity has in determining structure through torsional preferences, and find that common noncovalent locks increase aromaticity near planar torsional configurations. Ultimately, we identify hyperconjugation as the key stabilizing interaction that modifies aromaticity and results in planar structures. Our systematic study explains the success of prevalent noncovalent locks in conjugated molecules and polymers and will aid in the design of improved materials for organic electronics.

\end{abstract}
